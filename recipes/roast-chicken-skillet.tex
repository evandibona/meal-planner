\section{Skillet Roasted Chicken}

\subsection{ Ingredients }

\begin{itemize}
  \item <3 sprigs> Rosemary, finely chopped, plus more for garnish
  \item <6 sprigs> Thyme, finely chopped, plus more for garnish
  \item <8 sprigs> Parsley, flat-leaf, finely chopped
  \item <4 tbs> Butter, at room temperature
  \item <5/8 tsp> Salt
  \item <3/8 tsp> Black Pepper, freshly ground
  \item <2 lbs> Chicken, (3 1/2 – to 4-pound) chicken rinsed and dried, wings snipped at the elbow
  \item <3 head of garlic> Garlic, split horizontally
  \item <3 tbs> EVOO
\end{itemize}

\subsection{ Directions }

\begin{enumerate}
  \item Preheat the oven to 450°F (232°C).

  \item In a small bowl, mash half the rosemary, thyme, and parsley with the softened butter with a fork. Season with 1/8 teaspoon of the salt and 1/8 teaspoon of the pepper. Use your fingertips to slide the herbed butter beneath the skin of the breasts, starting at the opening near the neck and sliding it as far as possible beneath the skin. Stuff the cavity with the remaining chopped herbs and 1 head of garlic. Season with 1/2 teaspoon of salt and 1/4 teaspoon of pepper.

  \item Heat the oil in a heavy ovenproof sauté pan over medium-high heat until hot but not smoking.

  \item If you wish to sear the bird prior to roasting, truss the chicken with butcher’s twine*. Place the bird on its side in the skillet, searing the leg and breast. Leave it untouched in the pan for at least 4 minutes, turning only when the bird is burnished brown. Turn to brown the other breast side, and then the top and bottom of the bird so that it is well-browned on all sides. Spoon off the excess fat.

  \item If you don't wish to sear the bird prior to roasting, place the chicken in the skillet, breast side up.

  \item If desired, strew a couple sprigs of rosemary and thyme in the pan and place the remaining 2 heads garlic around the bird. Transfer the skillet to the oven. Roast until the chicken is cooked through, basting occasionally with the pan juices, for about 40 minutes. The bird is done when when the juices run clear from the joint between leg and thigh. If the bid’s skin begins to burn, cover with an aluminum-foil tent. Let stand at least 10 minutes prior to carving.
\end{enumerate}

\hrule
* Trussing a chicken note: To hold its shape and allow for even stovetop browning, it's important to properly truss the bird. Its upper wings are held tight against the body and the legs are crossed and bound at what would be the ankles. Begin by rinsing and drying the chicken, and then snipping the wings off at the elbow joint; discard or freeze the wing tips to make a stock. Lay a piece of kitchen twine, about 36 inches long, across a cutting board and place the chicken in the middle of the string. Pull the string up on both sides so that it draws the wings up against the bird's body. Bring each side of the string down along the breasts and pull snugly so the string tucks between the breast and the leg on each side. Wrap each piece of the string around the opposite ankle, then pull the string taut so the ankles are crossed and closed over the cavity. Tie the string very lightly and snip off any excess. The bird should now be snug, tight, and roundish.

\textit{Recipe © 2003 Keith McNally.}
