return {
  n = "Carrot and Leek Soup",
  ingr = { 
    "Carrots" = { 32, 'oz' },
    "Leeks" = { 12, 'oz' }, 
    "Celery" = { 12, 'oz' },
    "Chicken Stock = { 5, 'cups' },
    "Ghee" = { 2, 'tbs' },
    "Sour Cream" = { 1, 'cup' }
    "Salt" = { 2, 'tsp' }
  }
}
%%%%%%%%%%%%%%%%%%%%%%%%%%
% Be able to calculate the nutrients in a complete serving of a dish. 

\section{ Carrot \& Leek Soup }

\subsection{ Ingedients }

\begin{itemize}
  \item 2 tablespoons ghee
  \item 3 medium leeks, white and light green parts only, sliced thin 
  \item 2 teaspoons fresh thyme leaves
  \item 2 teaspoons finely ground real salt
  \item 4 ribs of celery, coursely chopped
  \item 2 pounds of carrots, coursely chopped
  \item 5 cups of chicken stock
  \item 1 cup of sour cream
\end{itemize}

\subsection{ Directions }

\begin{enumerate}
  \item Melt the ghee in the bottom of a heavy stock pot, over medium heat. Drop in the leeks and thyme leaves. Sprinkle them with salt, stir, and then cover the pot. Allow the leeks to sweat in the hot fat about 5 minutes, or until they begin to turn tender.
  \item Dump the celery and carrots into the pot, and pour in the chicken stock. Increase the heat to medium-high. When the contents of the pot reach a boil, turn down the heat to medium, and simmer, covered, about 25 minutes. Or until the vegetables are fall-apart tender. 
  \item Turn off the heat, and stir in the sour cream. Puree with an immersion blender until perfectly smooth, and then ladle into bowls to serve hot. Store any leftovers in a sealed container in the fridge for up to 5 days. 
\end{enumerate}

\subsection{ Variations }
\paragraph{Make it dairy-free} by swapping olive oil for the ghee and blending soaked cashew nuts with a squeeze of lemon in place of the sour cream. 
\paragraph{Swap shallots for the leeks,} and add lemongrass and ginger in place of fresh thyme. Swap the sour cream for full-fat coconut milk. 
\paragraph{Top the soup} with toasted mustard seeds and microgreens or with toasted pumpkin seeds. 


